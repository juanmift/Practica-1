\documentclass[11pt]{article}
    \title{\textbf{Práctica 1, ejercicio 2}}
    \author{Juan Miguel Fernández Tejada}
    \date{25/10/2022}
    
    \addtolength{\topmargin}{-3cm}
    \addtolength{\textheight}{3cm}
\begin{document}

\maketitle
\thispagestyle{empty}

\section{Within the folder “files”, find a TEX file in whose content appears the string
\\usepackage\{amsthm, amsmath\}. Note: use grep and escape the special
characters. Complete the proof and answer the question.}
Si $\alpha,\beta,\gamma$ son expresiones regulares entonces se cumple:
  \begin{equation}
  (\alpha+\beta)\gamma=\alpha\gamma+\beta\gamma
  \end{equation}
Usando las reglas de la definición tenemos que:

$$\mathcal{L}(((\alpha+\beta)\gamma))=\mathcal{L}((\alpha+\beta))\mathcal{L}(\gamma)=(\mathcal{L}(\alpha)\cup \mathcal{L}(\beta))\mathcal{L}(\gamma)=\mathcal{L}(\alpha)\mathcal{L}(\gamma)\cup \mathcal{L}(\beta)\mathcal{L}(\gamma)=$$\\
$$\mathcal{L}(\alpha\gamma)+\mathcal{L}(\beta\gamma)=\mathcal{L}(\alpha\gamma+\beta\gamma)$$\\\\
Consideremos $L=\{w\in \{a,b\}^* : w \textnormal{ no termina en } ab\}$. Un expresión regular que genera L es: \\
\[\epsilon + (a + b) + (a + b)^*(aa + ba + bb)\]
Explicación:
\begin{itemize}
\item $\epsilon$ genera a la cadena vacía.
\item $(a + b)$ genera a las cadenas de longitud igual a uno.
\item $(a + b)^*(aa + ba + bb)$ genera a cualquier cadena que tenga longitud mayor o igual a dos y que tenga cualquier terminación posible excepto $ab$.
\end{itemize}
\end{document}

